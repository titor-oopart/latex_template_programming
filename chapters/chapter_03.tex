\chapter{Introduccion a Jenkins}
El entendimiento de continous integration fue exitoso, y tenemos comparado muchas herramientas. diferentes herramientas estan disponibles en el mercado con diferentes tipos de funcionalidades y licenciamientos. En este capitulo, entenderemos como empezamos en Jenkins como una herramienta de continuous integration y las diferentes funcionalidades que jenkins provee. Tambien, haremos una comparacion detallada de diferentes herramientas de continuous integration.
\subsection*{Estructura}
Los siguientes temas se discutiran en este capitulo:
\begin{itemize}
  \item Panorama de herramienta para la integracion continua
  \item Por que Jenkins
  \item Instalacion de Jenkins en diferentes plataformas
  \item Que son plugins, y por que los necesitamos?
  \item Authenticacion y autorizacion dentro de Jenkins
  \item Jenkins Pipeline
  \item Librerias compartidas
  \item Que si el servidor es eliminado
  \item Maestro-esclavo Arquitectura
  \item Configuracion de Herramienta global
\end{itemize}
\subsection*{Objetivos}
Despues estudiar este capitulo, deberian estar disponible para decidir juntos escoger jenkins como una herramienta CI/CD. Deberias tambien ser capas de configurar diferentes componentes de jenkins y customizarlo para encapsularlo tus necesidades, 
\subsection{Panorama de herramientas}

